\documentclass[12pt]{report}
\usepackage{graphicx} % Required for inserting images
\usepackage{amsmath}
\usepackage{geometry}
\usepackage{ragged2e}
\usepackage{background}
\usepackage{indentfirst}
\usepackage{enumitem}
\usepackage{varwidth}
\usepackage{tasks}
\usepackage[portuguese]{babel}
\usetikzlibrary{calc}

\backgroundsetup{angle = 0, scale = 1, vshift = -2ex,
  contents = {\tikz[overlay, remember picture]
    \draw [rounded corners = 0pt, line width = 1pt,
           color = black, fill = white, double = blue!10]
           ($(current page.north west)+(1cm,-1cm)$)
           rectangle ($(current page.south east)+(-1,1)$);}}

\geometry{
 a4paper,
 total={170mm,257mm},
 left=20mm,
 top=25mm,
 }



\title{\textbf{\Huge Cálculo Diferencial e Integral}\\
Licenciatura em Engenharia Informática e de Computadores}
\author{Rafael Pereira}
\date{9 de Novembro, 2024}

\begin{document}

\maketitle
\tableofcontents

\chapter{Equações e Inequações Reais}
\section{Equações}
\vspace{2mm}

\large Equações são expressões matemáticas que mostram 
a igualdade entre dois termos ou expressões. 
Elas contêm variáveis, números e operações, 
como adição, subtração, multiplicação ou divisão. 
Uma equação estabelece uma relação que pode ser 
resolvida para encontrar o valor das variáveis. \par
\vspace{2mm}


\begin{center}
  \Large
  $ P_u(x) = a_0x^u + a_1x^{u-1} + \dots + a_{u-1}x + a_u \hspace{5mm} (a_0 \ne 0) $
  \vspace{5mm}

  $ Q_n(x) = b_0x^n + b_1x^{n-1} + \dots + b_{n-1}x + b_n \hspace{5mm} (b_0 \ne 0) $
  \vspace{5mm}

  $\frac{P_u(x)}{Q_n(x)} = 0 ; $
  $ \Leftrightarrow $
  $ \left\{ \begin{array}{rcl}
    P_u(x) = 0 \\
    Q_n(x) = 0
  \end{array}\right.$
\end{center}
\newpage


\section{Inequações}
\vspace{2mm}

\large Inequações são relações matemáticas que 
expressam desigualdade entre dois termos, utilizando 
símbolos como maior, menor, maior ou igual, e menor ou 
igual ($ > , < , \ge , \le  $). \par
\vspace{2mm}

\begin{center}
  \begin{varwidth}{\textwidth}
    \begin{enumerate}[label=\Roman*.]
      \item 
        $\frac{P_u(x)}{Q_n(x)} > 0 
        \Leftrightarrow
        \left\{ 
          \begin{array}{rcl}
            P_u(x) < 0 \\
            Q_n(x) < 0
          \end{array}
        \right.
        \vee
        \left\{ 
          \begin{array}{rcl}
          P_u(x) > 0 \\
          Q_n(x) > 0
          \end{array}
        \right.$ \vspace{2mm}

      \item 
        $\frac{P_u(x)}{Q_n(x)} \ge 0 
        \Leftrightarrow
        \left\{ 
          \begin{array}{rcl}
            P_u(x) \ge 0 \\
            Q_n(x) > 0
          \end{array}
        \right.
        \vee
        \left\{ 
          \begin{array}{rcl}
          P_u(x) \le 0 \\
          Q_n(x) < 0
          \end{array}
        \right.$ \vspace{2mm}

      \item 
        $\frac{P_u(x)}{Q_n(x)} < 0 
        \Leftrightarrow
        \left\{ 
          \begin{array}{rcl}
            P_u(x) < 0 \\
            Q_n(x) > 0
          \end{array}
        \right.
        \vee
        \left\{ 
          \begin{array}{rcl}
          P_u(x) > 0 \\
          Q_n(x) < 0
          \end{array}
        \right.$ \vspace{2mm}

      \item 
        $\frac{P_u(x)}{Q_n(x)} \le 0 
        \Leftrightarrow
        \left\{ 
          \begin{array}{rcl}
            P_u(x) \ge 0 \\
            Q_n(x) < 0
          \end{array}
        \right.
        \vee
        \left\{ 
          \begin{array}{rcl}
          P_u(x) \le 0 \\
          Q_n(x) > 0
          \end{array}
        \right.$ \vspace{2mm}


    \end{enumerate}
  \end{varwidth}
\end{center}

\section{Exemplo}

  \centering\Large
  $\frac{x^2-1}{2x-3} \ge x-4 \Leftrightarrow$
  \vspace{5mm}

  $\frac{x^2-1}{2x-3} - (x-4) \ge 0 $
  \vspace{5mm}

  $\Leftrightarrow\frac{x^2-1-(2x-3)(x-4)}{2x-3} \ge 0$
  \vspace{5mm}

  $\Leftrightarrow\frac{x^2-1-2x^2+8x+3x-12}{2s-3} \ge 0$
  \vspace{5mm}

  $\Leftrightarrow\frac{-x^2+11x-13}{2x-3} \ge 0$
  \vspace{5mm}

  $\Leftrightarrow\left\{\begin{array}{rcl}
    -x^2+11x-13 \ge 0\\
    2x-3 > 0
    \end{array}\right.
    \vee
    \left\{\begin{array}{rcl}
      -x^2+11x-13 \le 0 \\
      2x-3 < 0
      \end{array}\right.$
  \vspace{5mm}

  $\Leftrightarrow\left\{\begin{array}{rcl}
    -x^2+11x-13 \ge 0\\
    2x-3 > 0
    \end{array}\right.$
  \vspace{5mm}

  $\Leftrightarrow\left\{\begin{array}{rcl}
    x = \frac{11 \pm \sqrt{121-52}}2\\
    x \ge \frac{3}2
    \end{array}\right.$
  \vspace{5mm}

  $\Leftrightarrow\left\{\begin{array}{rcl}
    x = \frac{11 \pm \sqrt{121-52}}2\\
    \hrulefill
    \end{array}\right.$
  \vspace{5mm}
  
  \justifying

\section{Exercícios (com módulo)}

\centering
\begin{enumerate}
  \item \centering\Large $\left|4-5x\right| \ge 1  \Leftrightarrow\left\{\begin{array}{rcl}
    4-5x \ge 0 \\
    4-5x \ge 0
    \hrulefill
    \end{array}\right.$
    \vspace{5mm}
\end{enumerate}
\justifying

\end{document}